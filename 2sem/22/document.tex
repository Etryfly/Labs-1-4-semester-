\documentclass[11pt,a5paper,top=10mm]{book}
\usepackage[left=1cm,right=5mm,
top=5mm,bottom=5mm,bindingoffset=0cm]{geometry}
\usepackage[main=russian,english]{babel}
\usepackage[utf8]{inputenc}
\renewcommand{\baselinestretch}{0.8}


\pagestyle{empty}
\usepackage{amsmath}
\usepackage{amsfonts}
\usepackage{amssymb}
\usepackage{graphicx}

\begin{document}
	
	 \begin{scriptsize}
		264	
	  СИСТЕМЫ ОБЫКНОВЕННЫХ ДИФФЕРЕНЦИАЛЬНЫХ УРАВНЕНИЙ
	 [гл.\_VII
	 \end{scriptsize}
	 
	\par
	\medskip
	Ясно значение множителя C$_j$
	мы знаем, что если систему частных решений помножить на одно и то же произвольное постоянное, то получаем опять решение системы однородных линейных уравнений. Применяя проведенные рассуждения ко всем корням $\lambda_1$,$\lambda_2$,...,$\lambda_n$ характеристического уравнения, мы получим n частных решений вида (22) для j=1, 2,..., n.
	\par
	
	После этого мы можем написать полное решение системы (18) в виде:
	\par
	\hspace{20mm}
	$y_1=C_1y_1^{(1)}+C_2y_1^{(2)}+...+C_ny_1^{(n)},$
	\par
	\hspace{20mm}
	$y_2=C_1y_2^{(1)}+C_2y_2^{(2)}+...+C_ny_2^{(n)},$
	\par


		\hspace{20mm} .\ .\ .\ .\ .\ .\ .\ .\ .\ .\ .\ .\ .\ .\ .\ .\ .\ .\ .\ .\ .\ .\ .\ .\ .\ .\ 
		\par
	\hspace{20mm} .\ .\ .\ .\ .\ .\ .\ .\ .\ .\ .\ .\ .\ .\ .\ .\ .\ .\ .\ .\ .\ .\ .\ .\ .\ .\ 
	
	\par
	\hspace{20mm}
	$y_n=C_1y_n^{(1)}+C_2y_n^{(2)}+...+C_ny_n^{(n)}.$
	\par
	Примечание 1. Если коэффициенты уравнения действительны, а некоторые корни характеристического уравнения окажутся мнимыми, то они будут входить попарно сопряженными, например
	\par
	\hspace{25mm}
	$\lambda_1=\alpha+\beta i$,     $\lambda_2=\alpha-\beta i.$
	\newline
	Соответствующее решения будут иметь вид:
	\par
	\hspace{10mm}
	$\lambda_j^{(1)}=k_j^{(1)}e^{(\alpha+\beta)x}$, $y_j^{(1)}=k_j^{(2)}e^{(\alpha-\beta i)x}$,
	$(j=1, 2,..., n)$.
	\par
	Коэффициенты $k_j^{(1)}$ и $k_j^{(2)}$ тоже окажутся комплексными сопряженными, если взять их равными минорами одной и той же строки детерминантов $\Delta(\alpha+\beta i)$ и $\Delta(\alpha-\beta i)$. Легко убедиться, что корням $\lambda=\alpha \pm \beta i$ будут соответствовать две системы решений, соответствующих действительной и мнимой части $\lambda_j^{(1)}$ и $\lambda_j^{(2)}$, вида: 
	\newline
	$\widetilde{y}_j^{(1)}=e^{\alpha x} (l_j^{(1)} \cos{\beta x} - l_j^{(2)} sin{\beta x} ),\;
	 \widetilde{y}_j^{(2)}=e^{\alpha x} (l_j^{(1)} \sin{\beta x} + l_j^{(2)} cos{\beta x} ),\;$где $l_j^{(1)}$ и $l_j^{(2)}$ - действительные числа $k_j^{(1)}=l_j^{(1)}+il_j^{(2)},\; k_j^{(2)}=l_j^{(1)}-il_j^{(2)}.$
	 \par
	 Пример 5. $\dfrac{dy}{dx}+7y-z=0, \; \dfrac{dz}{dx}+2y+5z=0$. Ищем решение в виде $y=\gamma_1 e^{\lambda x}, z=\gamma_2e^{\lambda x};$ подставляя в заданную систему, получаем уравнения:
	 \par 
	 \hspace{10mm}
	 $\gamma_1 (\lambda + 7) - \gamma_2 = 9,\; 2\gamma_1+(\lambda+5)\gamma_2=0.$
	 \newline 
	 Условие их совместности дает характеристическое уравнение
	 \par 
	 $\begin{vmatrix}
		 \lambda+7& -1\\
		 2& \lambda+5
	 \end{vmatrix}$
	 , или $\lambda^2+12\lambda+37=0$
	 \par
	 Корни характеристического уравнения суть: $\lambda_1=-6+i, \; \lambda_2=-6-i.$
	 Подставляя первый из этих корней в систему для определения $\gamma_1$ и $\gamma_2,$ получаем два уравнения:
	 \par
	 $\gamma_1(1+i)-\gamma_2=0, \; 2\gamma_1+(-1+i)\gamma_2=0$,
	 \par из которых одно является следствием другого. Мы можем взять $k_1^{(1)}=1, k_2^{(1)}=1+i$. Первая система частных решений есть 
	 \par 
	 $y_1^{(1)}=e^{(-6+i)x}, \; y_2^{(1)}=(1+i)e^{(-6+i)x}$.
	 
	 \newpage
		$\S$
	  \hspace{1mm} 2]
	  \hspace{5mm}
	 Системы линейных дифференциальных уравнений \hspace{10mm} 265
	 
	 \medskip
	 Аналогично, подставляя корень $\lambda_1=-6-i$, найдем вторую систему частных решений:
	 \par
	 \hspace{10mm}
	 $y_1^{(2)}=e^{(-6-i)x}, \; y_2^{(2)}=(1-i)e^{(-6-i)x}$.
	 \par 
	 Беря в качестве новой фундаментальной системы решения
	
	\hspace{10mm}
	$ \overline{y}_i^{(1)}=\dfrac{y_i^{(1)}+y_i^{(2)}}{2}, \; \overline{y}_i^{(2)}=\dfrac{y_i^{(1)}-y_i^{(2)}}{2i} \; (i=1,2),$
	
	находим: 
	\par
	\hspace{3mm}
	$
	\overline{y}_1^{(1)}=e^{-6x} \cdot cosx; \;
	\overline{y}_1^{(2)}=e^{-6x}\cdot sinx; \;
	\overline{y}_2^{(1)}=e^{-6x}\cdot (cosx-sinx), \; $
	{\centering
		$
		 \overline{y}_2^{(2)}=e^{-6x} \cdot(cosx+sinx).
		$
		\par
	}	
	Общим решением будет:
	\par
	\hspace{10mm}
	$y_1=e^{-6x}(C_1 cosx + C_2 sinx),$
	\par
	\hspace{10mm}
	$
	y_2=e^{-6x} [(C_1+C_2)cosx+ (C_2 - C_1) sinx]. 
	$
	\par
	
	Примечание 2. Полученные нами n решений (22) являются линейно независимыми. В самом деле, рассмотрим таблицу (12); в нашем случае
	$y_i^{(j)}=k_i^{(j)}e^{\lambda}j^x.$
	Допустим, что в силу второго определения линейной зависимости выполняются соотношения (15), причем не все $\alpha_j=0$. С другой стороны, в каждой строке системы, например j-й, найдется коэфициент $k_i^{(j)} \neq 0$, иначе j-е частное решение было бы тривиальным.
	\par 
	В силу допущения мы имеем:
	\par
	$\alpha_1 k_i^{(1)} e^{\lambda_1 x} + \alpha_2 k_i^{(2)} e^{\lambda_2 x} + \dots + \alpha_j k_i^{(j)} e^{\lambda_j x} + \dots + \alpha_n k_i^{(n)} e^{\lambda_n x} = 0$
	\par
	Так как по доказанному в VI главе, функции $e^{\lambda} j^x$ линейно-независимы (j=1,2,..., n), то все коэфициенты в последнем соотношении равны нулю, в частности $\alpha_jk_i^{(j)}=0$.
	\newline
	В силу условия не все $k_i^{(j)} ( i= 1, 2,..., n)$, равны нулю, следовательо $\alpha_j=0.$
	\par 
	Это рассуждение применимо ко всем значениям j=1, 2,..., n, таким образом все $\alpha_j$ равны нулю. Полученное противоречие доказывает линейную независимость решений (22).
	\par
	
	2) Среди корней уравнения (21) есть кратные. Пусть $\lambda_1$ есть m-кратный корень характеристического уравнения. В таком случае значение m-й производной $\Delta (\lambda), \Delta^{(m)} (\lambda_1) \neq 0$, и рассуждение, аналогичное предыдущему, показывает, что среди миноров порядка m детерминанта $\Delta (\lambda)$ по крайней мере один отличен от нуля при $\lambda=\lambda_1$. Отсюда следует, что для ранга r матрицы $\textbf{M} (\lambda)$, при $\lambda=\lambda_1$ имеет место неравенство: $r \ge n-m$. В таком случае система линейных алгебраических уравнений (20) сводится к r независимым уравнениям. Из теории линейных уравнений известно, что тогда в общем решении системы (20) n=r неизвестных остаются произвольными, пусть это будут $\gamma_1 = C_1, \gamma_2 = C_2,...,\gamma_{n-r} = C_{n-r}$; остальные r неизвестных $\gamma_{n-r+1}, \gamma_{n-r+2},...,\gamma_{n}$ выразятся в виде линейных форм относительно $C_1, C_2,...,C_{n-r}$, пусть эти выражения будут:
	\par
	{\centering
		\hspace{15mm}
	$\gamma_j=k_j^{(1)}C_1+k_j^{(2)} C_2+...+k_j^{(n-r)} C_{n-r}.$	
	\newline
	$(j=n-r+1, n-r+2,...,n).$
\par} 
\end{document}
\\